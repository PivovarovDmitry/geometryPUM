%        File: 5.tex
%     Created: чт апр 09 08:00  2020 M
% Last Change: чт апр 09 08:00  2020 M
%
\documentclass[geometry,a5paper]{pum}
\listnumber{5}
\date{09.04.20}
\classname{10-3}
\lesson{13:30-15:20}
\begin{document}

Повторяем определения и термины:
\begin{itemize}
  \item выпуклый четырехугольник 
  \item площадь произвольного четырехугольника
  \item правильный многоугольник
  \item расстояние от точки до прямой
  \item правильная призма
  \item угол между прямыми в пространстве
  \item правильная пирамида
  \item площадь трапеции
  \item равнобедренная трапеция и ее свойства
  \item касание окружности и прямой
\end{itemize}


\begin{exercises}
  \begin{question}
    %\textcolor{darkcolortheme}{[Куланин, 10.7.2.]}
    В выпуклом четырехугольнике длины диагоналей 2 см и 4 см. Найти площадь четырехугольника, зная, что длины отрезков, соединяющих середины противоположных сторон, равны.
  \end{question}
  \begin{question}
    %\textcolor{darkcolortheme}{[Куланин, 10.7.4.]}
    Дан правильный 30-угольник $A_1 A_2 \ldots A_{30}$ с центром $O$. Найти уголь между прямыми $OA_3$ и $A_1 A_4$.
  \end{question}
  \begin{question}
    %\textcolor{darkcolortheme}{[Куланин, 10.7.13.]}
    Точка, лежащая внутри угла в $60^\circ$, удалена от его сторон на расстояния $a$ и $b$. Найти расстояние от этой точки до вершины угла.
  \end{question}
  \begin{question}
    %\textcolor{darkcolortheme}{[решуегэ.рф, 515782]}
    Дана треугольная призма $DEFD_1E_1F_1$. Она правильная с ребром 1.
    \begin{enumerate}[nosep,label=\asbuk*), ref=\asbuk*]
    \item Докажите, что прямая $DE_1$ параллельна прямой, проходящей через середины отрезков $DF$ и $EF_1$.
    \item Найдите косинус угла между прямыми $DE_1$ и $EF_1$.
  \end{enumerate}
%%    В правильной треугольной пирамиде $SABC$ с основанием $ABC$ известны ребра $AB=20\sqrt{3}$, $SC=29$.
%%    \begin{enumerate}[nosep,label=\asbuk*), ref=\asbuk*]
%%    \item Докажите, что $AS\perp BC$.
%%    \item Найдите угол, образованный плоскостью основания и прямой, проходящей через середины ребер $AS$ и $BC$.
%%  \end{enumerate}
\end{question}
  \begin{question}
    %\textcolor{darkcolortheme}{[решуегэ.рф, 523401]}
    Дана правильная треугольная пирамида $MABC$. Все ее боковые рёбра равны по 50. Сторона основания равна 60. Точки $G$ и $F$ делят стороны основания $AB$ и $AC$ в пропорции $AG:GB=AF:FC=1:5$.
    \begin{enumerate}[nosep,label=\asbuk*), ref=\asbuk*]
    \item Докажите, что сечение плоскостью $MGF$ -- равнобедренный треугольник.
    \item Найдите площадь этого сечения.
  \end{enumerate}
\end{question}
  \begin{question}
    %\textcolor{darkcolortheme}{[Куланин, 10.12.2.]}
    Длины боковых сторон $AB$ и $CD$ трапеции $ABCD$ равны соотвественно 8 см и 10 см, а длина основания $BC$ равна 2 см. Биссектриса угла $ADC$ проходит через середину стороны $AB$. Найти плодащь трапеции.
  \end{question}
  \begin{question}
    %\textcolor{darkcolortheme}{[Куланин, 10.12.17.]}
    В равнобедренной трапеции $ABCD$ длина основания $AD$ равна 14, а длина основания $BC$ -- 2. Окружность касается сторон $AB$, $BC$ и $CD$, причем боковая сторона делится точкой касания в отношении 1:9, считая от меньшего основания. Найти радиус окружности.

  \end{question}
\end{exercises}


\end{document}


