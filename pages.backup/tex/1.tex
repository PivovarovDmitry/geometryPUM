%        File: 1.tex
%     Created: вс мар 22 10:00  2020 M
% Last Change: вс мар 22 10:00  2020 M
%
\documentclass[geometry,a5paper]{pum}
\listnumber{1}
\date{19.03.20}
\classname{10-3}
\lesson{Д/З }
\begin{document}

Повторяем определения и термины:
\begin{itemize}
  \item правильный треугольник
  \item высота треугольника
  \item пирамида
  \item основание пирамиды
  \item бековое ребро пирамиды
  \item высота пирамиды
  \item перпендикулярность прямой и плоскости
  \item расстояние между прямыми
\end{itemize}

\begin{exercises}
  \begin{question}
    \textcolor{darkcolortheme}{[Куланин, 10.1.41]}
  Дан треугольник $ABC$, в котором $AC=5$, $AB=6$, $BC=7$. Биссектриса угла $C$ пересекает сторону $AB$ в точке $D$. Определить площадь треугольника $ADC$. 
  \end{question}
  \begin{question}
    \textcolor{darkcolortheme}{[Куланин, 10.2.15]}
    Основание равнобедренного треугольника $\sqrt{32}$, медиана боковой стороны 5. Найти длины боковых сторон.
  \end{question}
  \begin{question}
    Основание пирамиды $PABC$ -- правильный треугольник $ABC$, сторона которого равна 16, боковое ребро $PA$ -- $8\sqrt{3}$. Высота пирамиды $PH$ делит высоту $AM$ треугольника $ABC$ пополам. Через вершину $A$ проведена плоскость, перпендикулярная прямой $PM$ и пересекающая прямую $PM$ в точке $K$.
    \begin{enumerate}[label=\asbuk*), ref=\asbuk*]
    \item Докажите, что плоскость делит высоту $PH$ пирамиды $PABC$ в отношении 2:1, считая от вершины $P$.
    \item Найдите расстояние между прямыми $PH$ и $CK$.
  \end{enumerate}
\end{question}
\end{exercises}


\end{document}


