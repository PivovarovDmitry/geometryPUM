%        File: 5.tex
%     Created: чт апр 09 08:00  2020 M
% Last Change: чт апр 09 08:00  2020 M
%
\documentclass[geometry,a5paper]{pum}
\listnumber{6}
\date{14.04.20}
\classname{10-3}
\lesson{12:35-14:15}
\begin{document}

Повторяем определения, термины и свойства:
\begin{itemize}
  \item трапеция;
  \item площадь трапеции;
  \item плодащь треугольника;
  \item окружность, вписанная в ромб;
  \item окружность, вписанная в трапецию;
\end{itemize}


\begin{exercises}
  \begin{question}
    %\textcolor{darkcolortheme}{[Куланин, 10.8.1.]}
    Дано: $ABCD$ -- трапеция, $BC\parallel AD$, $E\in AB$, $AE=BE$.

    Доказать: $S_{\Delta ECD}=\frac{1}{2}S_{ABCD}$.
  \end{question}
  \begin{question}
    %\textcolor{darkcolortheme}{[Куланин, 10.8.4.]}
    Дано: $ABCD$ -- трапеция, $AB=CD$, $AD=a$, $BC=b$.

    Доказать: $d=\sqrt{ab}$, где $d$ -- диаметри вписанной в трапецию окружности.
  \end{question}
  \begin{question}
    %\textcolor{darkcolortheme}{[Куланин, 10.13.3.]}
    Дано: $ABCD$ -- ромб, $\angle B>90^\circ$, $BM\perp AC$, $BN\perp CD$, $\angle ABC = 2\arctg 2$, $r=1$ -- радиус окружности, вписанной в $BMDN$.

    Найти: $AB$.
  \end{question}

  \begin{question}
    В правильной четырехугольной призме $ABCDA_1 B_1 C_1 D_1$ на ребре $AA_1$ взята точка $M$ так, что $AM : MA_1 = 2 : 3$.
    \begin{enumerate}[label=\asbuk*)]
    \item Постройте сечение призмы плоскостью, проходящей через точки $D$ и $M$ параллельно диагонали основания $AC$.
    \item Найдите угол между плоскостью сечения и плоскостью основания, если $AA_1 = 65$, $AB=4$.
    \end{enumerate}
  \end{question}

  \begin{question}
    Дан куб $ABCDA_1 B_1 C_1 D_1$.
    \begin{enumerate}[label=\asbuk*)]
    \item Докажите, что прямая $B_1D$ перпендикулярна плоскости $A_1 BC_1$.
    \item Найдите угол между плоскостями $AB_1 C_1$ и $A_1BC_1$.
    \end{enumerate}
  \end{question}
\end{exercises}
\end{document}


