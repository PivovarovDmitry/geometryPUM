%        File: 1.tex
%     Created: вс мар 22 10:00  2020 M
% Last Change: вс мар 22 10:00  2020 M
%
\documentclass[geometry,a5paper]{pum}
%\usepackage{pst-all}
%\usepackage{pstricks-add}
%\usepackage[pdf]{pstricks}
%\usepackage{color}
%\usepackage{graphics}
\usepackage[ddmmyy]{datetime}
\renewcommand{\dateseparator}{.}

\listnumber{4}
%\date{07.04.20}
\date{\today}
\classname{10-3}
\lesson{12:35-14:15}
\begin{document}

%\def\moveto(#1,#2){%
  \pgfpathmoveto{\pgfpoint{#1 pt}{#2 pt}}}

\def\curveto(#1,#2)(#3,#4)(#5,#6){%
  \pgfpathcurveto{\pgfpoint{#1 pt}{#2 pt}}{\pgfpoint{#3 pt}{#4 pt}}{\pgfpoint{#5 pt}{#6 pt}}}

\def\lineto(#1,#2){%
  \pgfpathlineto{\pgfpoint{#1 pt}{#2 pt}} }

\let\psset=\tikzset
\def\pscustom[#1]{\tikzset{#1}}
\tikzset{
  xunit/.style={x={(#1,0)}},
  yunit/.style={y={(0,#1)}},
  runit/.style={},
  linecolor/.style={color=#1},
  linewidth/.style={line width=#1},
}

\def\pspicture(#1,#2){\tikzpicture}
\def\endpspicture{%
\pgfsetfillcolor{darkcolortheme}
\pgfusepath{fill}
\endtikzpicture}
\let\newpath=\relax
\let\closepath=\relax

%\begin{tikzpicture}[scale=0.1]
%\input{blue_bold.tex}
%\end{tikzpicture}

Повторяем определения и термины:
\begin{itemize}
  \item окружность
  \item диаметр
  \item радиус
  \item хорда
  \item сектор
  \item сегмент
  \item длина окружности
  \item площадь круга
  \item касающиеся окружности
  \item длина дуги окружности
\end{itemize}


\begin{exercises}
  \begin{question}
    %\textcolor{darkcolortheme}{[Куланин, 10.6.3.]}
    Две окружности радиуса $R$ и $\frac{R}{2}$ касаются друг друга внешним образом. Один из концов отрезка длины $2R$, образующего угол $30^\circ$ с линией центров, совпадает с центром окружности меньшего радиуса. Какая часть отрезка лежит вне окружностей?
  \end{question}
  \begin{question}
    %\textcolor{darkcolortheme}{[Куланин, 10.6.16.]}
    Найти сторону квадрата, вписанного в круг, площадь которого 64 см$^2$.
  \end{question}
  \begin{question}
    %\textcolor{darkcolortheme}{[Куланин, 10.6.26.]}
    В окружности пересекающиеся хорды $AB$ и $CD$ перпендикулярны, $AD=m$, $BC=n$. Найти диаметр окружности.
  \end{question}
  \begin{question}
    %\textcolor{darkcolortheme}{[решуегэ.рф, 505550]}
    Даны ребра $AB=20\sqrt{3}$ и $DC=29$ правильной треугольной пирамиды $DABC$. 
    \begin{enumerate}[nosep,label=\asbuk*), ref=\asbuk*]
    \item Перпендикулярны ли ребра $DA$ и $CB$?
    \item Прямая $l$ проходит через середины этих же ребер. Найти угол между прямой $l$ и плоскостью основания.
  \end{enumerate}
%%    В правильной треугольной пирамиде $SABC$ с основанием $ABC$ известны ребра $AB=20\sqrt{3}$, $SC=29$.
%%    \begin{enumerate}[nosep,label=\asbuk*), ref=\asbuk*]
%%    \item Докажите, что $AS\perp BC$.
%%    \item Найдите угол, образованный плоскостью основания и прямой, проходящей через середины ребер $AS$ и $BC$.
%%  \end{enumerate}
\end{question}
  \begin{question}
    %\textcolor{darkcolortheme}{[решуегэ.рф, 509121]}
    В пирамиде $DABC$ прямые, содержащие непересекающиеся боковое ребро и ребро основания перпендикулярны.
    \begin{enumerate}[nosep,label=\asbuk*), ref=\asbuk*]
    \item Постройте сечение плоскостью. Она параллельна этим ребрам и проходит через середину другого бокового ребра. Докажите, что это сечение -- прямоугольник.
    \item Найдите угол между диагоналями этого прямоугольника, если перпендикулярные ребра равны соответственно 30 и 16.
  \end{enumerate}
\end{question}

\subsubsection*{Задачи повышенной трудности}
  \begin{question}
    %\textcolor{darkcolortheme}{[Куланин, 10.10.2.]}
    Даны равносторонний треугольник со стороной $a$ и окружность, касающаяся одной из сторон треугольника и делящая вторую сторону на две равные части. Кроме того, известно, что центр окружности лежит на третьей стороне треугольника. Найти расстояние от центра окружности до ближайшей вершины треугольника.
  \end{question}
  \begin{question}
    %\textcolor{darkcolortheme}{[Куланин, 10.11.5.]}
    В прямоугольном треугольнике $ABC$ с гипотенузой $AB$ и площадью 30 точка $O$ -- центр вписанной окружности. Площадь треугольника $AOB$ равна 13. Найти длины сторон треугольника $ABC$.

  \end{question}
\end{exercises}


\end{document}


