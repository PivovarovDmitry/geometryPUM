%        File: pum.tex
%     Created: ср апр 15 02:00  2020 M
% Last Change: ср апр 15 02:00  2020 M
%
\documentclass[slidestop,xcolor=pst,dvips]{beamer}
\usepackage[utf8]{inputenc}
\usepackage[russian]{babel}
\usepackage{amsmath,amssymb}
\usepackage{pstricks}
\usepackage{pst-node}
\usepackage{pst-arrow}
\usepackage{pst-tools}
\usepackage{pst-plot}
\usepackage{pst-eucl}

\author{Пивоваров Д.Е.}
\title{Решение задач по геометрии}

\begin{document}
%\maketitle

\begin{frame}
\frametitle{Задача 10.8.1}
\vspace{-5cm}
\begin{figure}
  \begin{pspicture}(10,10)
  \pstGeonode[PosAngle={180,180,0,0},CurveType=polygon](0,0){A}(1,3){B}(4,3){C}(6,0){D}
  \pstMiddleAB[PosAngle={180}]{A}{B}{E}
  \psset{SegmentSymbol=MarkHash}
  \pstSegmentMark{A}{E}
  \pstSegmentMark{B}{E}
  \pstLineAB{E}{C}
  \pstLineAB{E}{D}
  \psset{linecolor=blue,CodeFig=false}
  \only<2-6>{
    \pstMiddleAB[PosAngle=60]{C}{D}{F}
    \psset{SegmentSymbol=MarkHashh,MarkAngle=-45}
    \pstSegmentMark{C}{F}
    \pstSegmentMark{D}{F}}
  \only<3->{
    %\psset{PointName=default,PointSymbol=default}
    \pstLineAB[nodesepB=-2.0]{E}{F}}
  \only<4-6>{
    \pstMarkAngle[linecolor=violet]{C}{F}{E}{}
    \pstMarkAngle[linecolor=violet]{D}{F}{D'}{}}
  \only<5->{
    \pstTranslation[PosAngle=120]{A}{E}{D}[D']
    \pstTranslation[PosAngle=120]{B}{E}{C}[C']
    \pstLineAB[PointName=$D'$,nodesep=-0.5]{D}{D'}
    \pstLineAB[nodesep=-0.5]{C}{C'}}
  \only<6>{
    %\psset{MarkAngleType=double}
    \pstMarkAngle[linecolor=green]{C'}{C}{F}{}
    \pstMarkAngle[linecolor=green]{D'}{D}{F}{}}
  \only<7->{
%  \pstTriangle(\pstAbscissa{C},\pstOrdinate{C}){C}(\pstAbscissa{C'},\pstOrdinate{C'}){C'}(\pstAbscissa{F},\pstOrdinate{F}){F}
%    \pstTriangle(\pstAbscissa{D},\pstOrdinate{D}){D}(\pstAbscissa{D'},\pstOrdinate{D'}){D'}(\pstAbscissa{F},\pstOrdinate{F}){F}
     k
    }
\end{pspicture}
\end{figure}
\end{frame}

\begin{frame}
\frametitle{Задача 10.8.1}
  \begin{columns}
    \begin{column}{0.49\textwidth}
      \begin{figure}
        \begin{pspicture}(10,10)
  \pstGeonode[PosAngle={180,180,0,0},CurveType=polygon](0,0){A}(1,3){B}(4,3){C}(6,0){D}
  \pstMiddleAB[PosAngle={180}]{A}{B}{E}
  \pstSegmentMark{A}{E}
  \pstSegmentMark{B}{E}
  \pstLineAB{E}{C}
  \pstLineAB{E}{D}
  \psset{linecolor=gray,CodeFig=false,PointName=none,PointSymbol=none,nodesep=-1.5}
  \pstMiddleAB{C}{D}{F}
  \pstLineAB{E}{F}
  \pstTranslation{B}{C}{A,B}[A1,B1]
  \pstTranslation{A}{D}{A,B}[A2,B2]
  \pstLineAB{A1}{B1}
  \pstLineAB{A2}{B2}
%  \pstGeonode[PointName=O,PosAngle=180]{IC_O}
\end{pspicture}
\end{figure}
\end{column}
    \begin{column}{0.49\textwidth}
      \begin{enumerate}
        \item Построим:
          \begin{gather*}
            EE'||AD \text{средняя линия трапеции}\\
            CC'||AB \\
            DD'||AB \\
          \end{gather*}
        \item $\Delta CC'E' \sim \Delta DD'E'$ по стороне и двум прилежащим гулам: $CE'=DE'$ (средняя линия трапеции), внетренние накрестлежащие и вертикальные.
        \item Площадь трапеции слагается из площадей двух параллелограммов.
        \item А сумма половинок этих параллелограммов являются суммой двух частей треугольника.
      \end{enumerate}
\end{column}
  \end{columns}
\end{frame}

\end{document}
