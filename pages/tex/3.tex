%        File: 1.tex
%     Created: вс мар 22 10:00  2020 M
% Last Change: вс мар 22 10:00  2020 M
%
\documentclass[geometry,a5paper]{pum}
\listnumber{3}
\date{26.03.20}
\classname{10-3}
\lesson{13:30-15:20 }
\begin{document}

Повторяем определения и термины:
\begin{itemize}
  \item определение равнобокой трапеции и ее свойства
  \item определение прямоугольника и его свойства
  \item определение ромба и его свойства
  \item прямая призма
  \item высота призмы
  \item угол между прямой и плоскостью
  \item теорема о трех перпендикулярах
  \item правильная пирамида
\end{itemize}


\begin{exercises}
  \begin{question}
    \textcolor{darkcolortheme}{[Куланин, 10.4.14.]}
    Периметр равнобедренной трапеции с острым углом $\alpha$ равен $p$. Высота трапеции равна $h$. Найти площадь этой тапеции.
  \end{question}
  \begin{question}
    \textcolor{darkcolortheme}{[Куланин, 10.5.3.]}
    В прямоугольнике $ABCD$ дано: $AB=a$, $AD=b$. найти на стороне $AB$ точку $E$, для которой $\angle CED=\angle AED$.
  \end{question}
  \begin{question}
    \textcolor{darkcolortheme}{[Куланин, 10.5.21.]}
    В ромбе $ABCD$ угол при вершине $A$ равен $\frac{\pi}{3}$. Точка $N$ делит сторону $AB$ в отношении $AN:BN=2:1$. Определить тангент угла $DNC$.
  \end{question}
  \begin{question}
    \textcolor{darkcolortheme}{[решуегэ.рф, 511424]}
    Основанием прямой призмы $ABCA_1B_1C_1$ является равнобедренный треугольник  $ABC, AB=AC=13, BC=24$. Высота призмы равна~5.
    \begin{enumerate}[nosep,label=\asbuk*), ref=\asbuk*]
    \item Докажите, что сечение призмы плоскостью, содержащей ребро $AA_1$ и перпендикулярной плоскости $BCC_1$, является квадратом.
    \item Найдите угол между прямой $A_1B$ и плоскостью $BCC_1$.  
  \end{enumerate}
\end{question}
  \begin{question}
    \textcolor{darkcolortheme}{[решуегэ.рф, 484659]}
    В правильной треугольной пирамиде $SABC$ с основанием $ABC$ известны ребра $AB=7\sqrt{3}$, $SC = 25$. $M$ -- середина ребра $SA$.
    \begin{enumerate}[nosep,label=\asbuk*), ref=\asbuk*]
    \item Докажите, что проекции точек $S$ и $M$ на плоскость основания делят высоту $AN$ треугольника $ABC$ на три равные части.
    \item Найдите угол, образованный плоскостью основания и прямой $MN$.
  \end{enumerate}
\end{question}

\subsubsection*{Задачи повышенной трудности}
  \begin{question}
    \textcolor{darkcolortheme}{[Куланин, 10.9.7.]}
    В треугольнике $ABC$ точка $E$ принадлежит медиане $BD$, причем $BE=3ED$. Прямая $AE$ пересекает сторону $BC$ в точке $M$. Найти отношение площадей треугольников $AMC$ и $ABC$. Или коротко:

    Дано: $\Delta ABC$, $BD$ -- медиана, $E\in BD$, $BE=3ED$, $AE \cap BC=M$.
    
    Найти: $S_{\Delta AMC}:S_{\Delta ABC}$.
  \end{question}
  \begin{question}
    \textcolor{darkcolortheme}{[Куланин, 10.9.31.]}
    Найти углы треугольника с единичным радиусом вписанной окружности, если известно, что длины его высот -- целые числа.
  \end{question}
\end{exercises}


\end{document}


