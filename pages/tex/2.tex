%        File: 1.tex
%     Created: вс мар 22 10:00  2020 M
% Last Change: вс мар 22 10:00  2020 M
%
\documentclass[geometry,a5paper]{pum}
\listnumber{2}
\date{24.03.20}
\classname{10-3}
\lesson{12:35-14:15 }
\begin{document}

Повторяем определения и термины:
\begin{itemize}
  \item формула нахождения площади круга
  \item свойства окружности, описанной вокруг треугольника
  \item свойство равнобочной трапеции
  \item свойства параллелепипеда
  \item правильная призма
  \item боковое ребро призмы
  \item основание призмы
  \item параллельность прямой и плоскости
  \item угол между плоскостями
\end{itemize}


\begin{exercises}
  \begin{question}
    \textcolor{darkcolortheme}{[Куланин, 10.3.7.]}
    Периметр прямоугольного треугольника равен 24 см, а площадь его равна 24 см$^2$. Найти площадь описанного круга.
  \end{question}
  \begin{question}
    \textcolor{darkcolortheme}{[Куланин, 10.3.49.]}
    В прямоугольном треугольнике биссектриса прямого угла делит гипотенузу на отверзки 3 см и 4 см. Найти плодащь треугольника. 
  \end{question}
  \begin{question}
    \textcolor{darkcolortheme}{[Куланин, 10.4.1.]}
    Площадь равнобочной трапеции равна $S$, угол между ее диагоналями, противолежащий боковой стороне, равен $\alpha$. Найти высоту трапеции.
  \end{question}
  \begin{question}
    \textcolor{darkcolortheme}{[решуегэ.рф, 511703]}
  Все рёбра правильной треугольной призмы $ABCA_1B_1C_1$ имеют длину 12. Точки $M$ и $N$ -- середины рёбер $AA_1$ и $A_1C_1$ соответственно.
    \begin{enumerate}[nosep,label=\asbuk*), ref=\asbuk*]
    \item Докажите, что прямые $BM$ и $MN$ перпендикулярны.
    \item Найдите угол между плоскостями $BMN$ и $ABB_1$.
  \end{enumerate}
\end{question}
  \begin{question}
    \textcolor{darkcolortheme}{[решуегэ.рф, 516880]}
    В параллелепипеде $ABCDA_1B_1C_1D_1$ точка $F$ середина ребра $AB$, а точка $E$ делит ребро $DD_1$ в отношении $DE : ED_1 = 6 : 1$. Через точки $F$ и $E$ проведена плоскость $\alpha$, параллельная прямой $AC$ и пересекающая диагональ $B_1D$ в точке $О$.
    \begin{enumerate}[nosep,label=\asbuk*), ref=\asbuk*]
    \item Докажите, что плоскость $\alpha$ делит диагональ $DB_1$ в отношении $DO : OB_1 = 2 : 3$.
    \item Найдите угол между плоскостью $\alpha$ и плоскостью $(ABC)$, если дополнительно известно, что $ABCDA_1B_1C_1D_1$ -- правильная четырехугольная призма, сторона основания которой равна 4, а высота равна 7.
  \end{enumerate}
\end{question}
  \begin{question}
    \textcolor{darkcolortheme}{[решуегэ.рф, 516399]}
    Дана правильная треугольная призма $ABCA_1B_1C_1$, у которой сторона основания равна 2, а боковое ребро равно 3. Через точки $A$, $C_1$ и середину $T$ ребра $A_1B_1$ проведена плоскость.
    \begin{enumerate}[nosep,label=\asbuk*), ref=\asbuk*]
    \item Докажите, что сечение призмы указанной плоскостью является прямоугольным треугольником.
    \item Найдите угол между плоскостью сечения и плоскостью $ABC$.
  \end{enumerate}
\end{question}
\end{exercises}


\end{document}


